\documentclass[conference]{IEEEtran}

% --- Paket Wajib ---
\usepackage{cite}
\usepackage{amsmath,amssymb,amsfonts}
\usepackage{algorithmic}
\usepackage{graphicx}
\usepackage{textcomp}
\usepackage{xcolor}
\usepackage{float}
\usepackage{hyperref}

% --- Judul & Penulis ---
\title{Implementasi Regresi Linear Berganda dengan Optimasi Gradient Descent untuk Prediksi Harga Properti}

\author{\IEEEauthorblockN{Muhamad Dimas Saputra}
\IEEEauthorblockA{\textit{Teknik Informatika/Sains dan Teknologi} \\
\textit{UNIVERSITAS ISLAM NEGERI SYARIF HIDAYATULLAH JAKARTA}\\
Tangerang Selatan, Indonesia \\
muhamad.dimas24@mhs.uinjkt.ac.id}
}

\begin{document}

\maketitle

\begin{abstract}
okr
\end{abstract}

% --- Kata Kunci ---
\begin{IEEEkeywords}
Prediksi Harga Rumah, Regresi Linear, Gradient Descent, Machine Learning, Metode Numerik.
\end{IEEEkeywords}

% --- BAB I ---
\section{Pendahuluan}
Sektor properti memainkan peran penting dalam perekonomian. Namun, prediksi harga yang akurat sulit dilakukan secara manual karena banyaknya variabel yang berpengaruh, seperti luas tanah, jumlah kamar, dan lokasi \cite{Hallan2025}.

Pendekatan \textit{Machine Learning} menawarkan solusi untuk masalah ini. Penelitian sebelumnya oleh Siregar et al. menunjukkan bahwa regresi linear mampu memberikan estimasi yang baik dengan memanfaatkan variabel fisik bangunan \cite{Siregar2023}. Dalam penelitian ini, penulis berfokus pada implementasi algoritma optimasi \textit{Gradient Descent} untuk menyelesaikan fungsi biaya secara numerik.

% --- BAB II ---
\section{Landasan Teori}

\subsection{Model Regresi Linear}
Secara matematis, prediksi harga rumah ($\hat{y}$) dimodelkan sebagai kombinasi linear dari fitur input ($x$). Persamaan hipotesis didefinisikan sebagai:
\begin{equation}
    h_\theta(x) = \theta_0 + \theta_1 x_1 + \theta_2 x_2 + \dots + \theta_n x_n
\end{equation}
Dimana $\theta$ adalah parameter bobot yang akan dipelajari oleh model.

\subsection{Fungsi Biaya (Cost Function)}
Untuk mengukur kinerja model, digunakan \textit{Mean Squared Error} (MSE). Fungsi biaya $J(\theta)$ bertujuan untuk meminimalkan selisih kuadrat antara prediksi dan nilai asli \cite{Ng2022}:
\begin{equation}
    J(\theta) = \frac{1}{2m} \sum_{i=1}^{m} (h_\theta(x^{(i)}) - y^{(i)})^2
    \label{eq:cost_function}
\end{equation}

\subsection{Algoritma Gradient Descent}
Parameter $\theta$ diperbarui secara iteratif untuk mencapai konvergensi global minimum menggunakan aturan pembaruan:
\begin{equation}
    \theta_j := \theta_j - \alpha \frac{\partial}{\partial \theta_j} J(\theta)
\end{equation}
Di mana $\alpha$ adalah \textit{learning rate} yang menentukan seberapa besar langkah yang diambil setiap iterasi.

% --- BAB III ---
\section{Metodologi}

\subsection{Pra-pemrosesan Data}
Sebelum dilakukan pelatihan, data mentah harus melalui tahap pra-pemrosesan. Salah satu tahapan krusial adalah standarisasi fitur (\textit{Feature Scaling}) menggunakan teknik \textit{Z-score normalization}:
\begin{equation}
    x' = \frac{x - \mu}{\sigma}
\end{equation}
Hal ini penting agar kontur fungsi biaya menjadi lebih simetris, sehingga \textit{Gradient Descent} dapat konvergen lebih cepat \cite{Hallan2025}.

% --- BAB IV ---
\section{Hasil dan Pembahasan}
Implementasi dilakukan menggunakan bahasa Python dengan pustaka NumPy. Grafik penurunan \textit{Loss} dapat dilihat pada Gambar \ref{fig:loss_graph}.

% Placeholder Gambar (Dibuat komentar agar tidak error jika file tidak ada)
\begin{figure}[htbp]
    \centering
    % \includegraphics[width=0.8\linewidth]{loss_plot.png} 
    \caption{Ilustrasi Grafik Penurunan Cost Function (Contoh)}
    \label{fig:loss_graph}
\end{figure}

% --- BAB V ---
\section{Kesimpulan}
Berdasarkan hasil eksperimen, algoritma Regresi Linear dengan optimasi Gradient Descent terbukti mampu memprediksi harga rumah dengan tingkat error yang dapat diterima.

% --- Daftar Pustaka ---
\bibliographystyle{IEEEtran}
\bibliography{references}

\end{document}